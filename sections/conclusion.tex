\section{Conclusion}
\setlength{\parindent}{20pt}
 \indent Measures in a flow rig ask theoretic knowledge. The specificity of the duct acoustic is the possibility to work with the modes because of the low amount of them. 
 
 Resolve the convective Helmholtz equation allows to describes these modes and their shapes. In the axial direction a positive and a negative wave can propagate, the modes have different shapes in the section. For the rectangular duct, the different waves numbers are link but the modes are independent in each direction. While for the annular case the circumferential and the radial modes are linked by a Bessel function.
The optimization of an acoustic liner is not easy for two main reasons: 
\begin{itemize}
    \item The dimension of the duct are very high compare to the dimensions of the liner cavities. The problem asks too many resources for numerical methods. The acoustic impedance has to be used to simplify the problem.
    \item The acoustic theory for lined wall in a circular duct is though. The optimum Tester and Cremer impedance is well determined to reduce the plane wave mode in a circular duct. In this master thesis, we tried to generalize for a higher mode and for the annular case 
\end{itemize}
\indent To get the acoustic properties some works have to be done into the time data. This post-treatment is based on models and adds some uncertainties into the final values.

Last but not least the measurement with flow must be done carefully. The noise created by the flow is important.

The first results show a high transmission loss. However the disparity of the 3 liners seems to be high. The future determination of the impedance will confirm if something is wrong. There were some manufactured defaults in the Chinese liners what can explain this disparity.
\clearpage

